% Options for packages loaded elsewhere
\PassOptionsToPackage{unicode}{hyperref}
\PassOptionsToPackage{hyphens}{url}
%
\documentclass[
]{article}
\usepackage{amsmath,amssymb}
\usepackage{iftex}
\ifPDFTeX
  \usepackage[T1]{fontenc}
  \usepackage[utf8]{inputenc}
  \usepackage{textcomp} % provide euro and other symbols
\else % if luatex or xetex
  \usepackage{unicode-math} % this also loads fontspec
  \defaultfontfeatures{Scale=MatchLowercase}
  \defaultfontfeatures[\rmfamily]{Ligatures=TeX,Scale=1}
\fi
\usepackage{lmodern}
\ifPDFTeX\else
  % xetex/luatex font selection
\fi
% Use upquote if available, for straight quotes in verbatim environments
\IfFileExists{upquote.sty}{\usepackage{upquote}}{}
\IfFileExists{microtype.sty}{% use microtype if available
  \usepackage[]{microtype}
  \UseMicrotypeSet[protrusion]{basicmath} % disable protrusion for tt fonts
}{}
\makeatletter
\@ifundefined{KOMAClassName}{% if non-KOMA class
  \IfFileExists{parskip.sty}{%
    \usepackage{parskip}
  }{% else
    \setlength{\parindent}{0pt}
    \setlength{\parskip}{6pt plus 2pt minus 1pt}}
}{% if KOMA class
  \KOMAoptions{parskip=half}}
\makeatother
\usepackage{xcolor}
\usepackage[margin=1in]{geometry}
\usepackage{color}
\usepackage{fancyvrb}
\newcommand{\VerbBar}{|}
\newcommand{\VERB}{\Verb[commandchars=\\\{\}]}
\DefineVerbatimEnvironment{Highlighting}{Verbatim}{commandchars=\\\{\}}
% Add ',fontsize=\small' for more characters per line
\usepackage{framed}
\definecolor{shadecolor}{RGB}{248,248,248}
\newenvironment{Shaded}{\begin{snugshade}}{\end{snugshade}}
\newcommand{\AlertTok}[1]{\textcolor[rgb]{0.94,0.16,0.16}{#1}}
\newcommand{\AnnotationTok}[1]{\textcolor[rgb]{0.56,0.35,0.01}{\textbf{\textit{#1}}}}
\newcommand{\AttributeTok}[1]{\textcolor[rgb]{0.13,0.29,0.53}{#1}}
\newcommand{\BaseNTok}[1]{\textcolor[rgb]{0.00,0.00,0.81}{#1}}
\newcommand{\BuiltInTok}[1]{#1}
\newcommand{\CharTok}[1]{\textcolor[rgb]{0.31,0.60,0.02}{#1}}
\newcommand{\CommentTok}[1]{\textcolor[rgb]{0.56,0.35,0.01}{\textit{#1}}}
\newcommand{\CommentVarTok}[1]{\textcolor[rgb]{0.56,0.35,0.01}{\textbf{\textit{#1}}}}
\newcommand{\ConstantTok}[1]{\textcolor[rgb]{0.56,0.35,0.01}{#1}}
\newcommand{\ControlFlowTok}[1]{\textcolor[rgb]{0.13,0.29,0.53}{\textbf{#1}}}
\newcommand{\DataTypeTok}[1]{\textcolor[rgb]{0.13,0.29,0.53}{#1}}
\newcommand{\DecValTok}[1]{\textcolor[rgb]{0.00,0.00,0.81}{#1}}
\newcommand{\DocumentationTok}[1]{\textcolor[rgb]{0.56,0.35,0.01}{\textbf{\textit{#1}}}}
\newcommand{\ErrorTok}[1]{\textcolor[rgb]{0.64,0.00,0.00}{\textbf{#1}}}
\newcommand{\ExtensionTok}[1]{#1}
\newcommand{\FloatTok}[1]{\textcolor[rgb]{0.00,0.00,0.81}{#1}}
\newcommand{\FunctionTok}[1]{\textcolor[rgb]{0.13,0.29,0.53}{\textbf{#1}}}
\newcommand{\ImportTok}[1]{#1}
\newcommand{\InformationTok}[1]{\textcolor[rgb]{0.56,0.35,0.01}{\textbf{\textit{#1}}}}
\newcommand{\KeywordTok}[1]{\textcolor[rgb]{0.13,0.29,0.53}{\textbf{#1}}}
\newcommand{\NormalTok}[1]{#1}
\newcommand{\OperatorTok}[1]{\textcolor[rgb]{0.81,0.36,0.00}{\textbf{#1}}}
\newcommand{\OtherTok}[1]{\textcolor[rgb]{0.56,0.35,0.01}{#1}}
\newcommand{\PreprocessorTok}[1]{\textcolor[rgb]{0.56,0.35,0.01}{\textit{#1}}}
\newcommand{\RegionMarkerTok}[1]{#1}
\newcommand{\SpecialCharTok}[1]{\textcolor[rgb]{0.81,0.36,0.00}{\textbf{#1}}}
\newcommand{\SpecialStringTok}[1]{\textcolor[rgb]{0.31,0.60,0.02}{#1}}
\newcommand{\StringTok}[1]{\textcolor[rgb]{0.31,0.60,0.02}{#1}}
\newcommand{\VariableTok}[1]{\textcolor[rgb]{0.00,0.00,0.00}{#1}}
\newcommand{\VerbatimStringTok}[1]{\textcolor[rgb]{0.31,0.60,0.02}{#1}}
\newcommand{\WarningTok}[1]{\textcolor[rgb]{0.56,0.35,0.01}{\textbf{\textit{#1}}}}
\usepackage{graphicx}
\makeatletter
\def\maxwidth{\ifdim\Gin@nat@width>\linewidth\linewidth\else\Gin@nat@width\fi}
\def\maxheight{\ifdim\Gin@nat@height>\textheight\textheight\else\Gin@nat@height\fi}
\makeatother
% Scale images if necessary, so that they will not overflow the page
% margins by default, and it is still possible to overwrite the defaults
% using explicit options in \includegraphics[width, height, ...]{}
\setkeys{Gin}{width=\maxwidth,height=\maxheight,keepaspectratio}
% Set default figure placement to htbp
\makeatletter
\def\fps@figure{htbp}
\makeatother
\setlength{\emergencystretch}{3em} % prevent overfull lines
\providecommand{\tightlist}{%
  \setlength{\itemsep}{0pt}\setlength{\parskip}{0pt}}
\setcounter{secnumdepth}{-\maxdimen} % remove section numbering
\ifLuaTeX
  \usepackage{selnolig}  % disable illegal ligatures
\fi
\usepackage{bookmark}
\IfFileExists{xurl.sty}{\usepackage{xurl}}{} % add URL line breaks if available
\urlstyle{same}
\hypersetup{
  pdftitle={Week 1 Exercises},
  pdfauthor={PJ Grant},
  hidelinks,
  pdfcreator={LaTeX via pandoc}}

\title{Week 1 Exercises}
\author{PJ Grant}
\date{October 21, 2024}

\begin{document}
\maketitle

Please complete all exercises below WITHOUT using any
libraries/packages.

\section{Exercise 1}\label{exercise-1}

\subsection{subheader}\label{subheader}

\subsubsection{heavily nested}\label{heavily-nested}

\paragraph{}\label{section}

Assign 10 to the variable x. Assign 5 to the variable y. Assign 20 to
the variable z.

\begin{Shaded}
\begin{Highlighting}[]
\NormalTok{x }\OtherTok{=} \DecValTok{10}
\NormalTok{y }\OtherTok{=} \DecValTok{5}
\NormalTok{z }\OtherTok{=} \DecValTok{20}
\end{Highlighting}
\end{Shaded}

\section{Exercise 2}\label{exercise-2}

Show that x is less than z but greater than y.

\textbf{Note: your output must be a SINGLE boolean, do not output a
boolean for each expression.}

\begin{Shaded}
\begin{Highlighting}[]
\NormalTok{z }\SpecialCharTok{\textgreater{}}\NormalTok{ x }\SpecialCharTok{\&}\NormalTok{ x }\SpecialCharTok{\textgreater{}}\NormalTok{ y}
\end{Highlighting}
\end{Shaded}

\begin{verbatim}
## [1] TRUE
\end{verbatim}

\section{Exercise 3}\label{exercise-3}

Show that x and y do not equal z.

\textbf{Note: your output must be a SINGLE boolean, do not output a
boolean for each expression.}

\begin{Shaded}
\begin{Highlighting}[]
\NormalTok{(x }\SpecialCharTok{\&}\NormalTok{ y) }\SpecialCharTok{!=}\NormalTok{ z}
\end{Highlighting}
\end{Shaded}

\begin{verbatim}
## [1] TRUE
\end{verbatim}

\section{Exercise 4}\label{exercise-4}

Show that the formula \texttt{x\ +\ 2y\ =\ z}.

\textbf{Note: your output must be a SINGLE boolean}

\begin{Shaded}
\begin{Highlighting}[]
\NormalTok{(x }\SpecialCharTok{+}\NormalTok{ (}\DecValTok{2} \SpecialCharTok{*}\NormalTok{ y)) }\SpecialCharTok{==}\NormalTok{ z}
\end{Highlighting}
\end{Shaded}

\begin{verbatim}
## [1] TRUE
\end{verbatim}

\section{Exercise 5}\label{exercise-5}

I have created a vector (test\_vector) of integers for you. Determine if
any of x, y, or z are in the vector.

\textbf{Note: your output must be a SINGLE boolean, do not output a
boolean for each expression.}

\begin{Shaded}
\begin{Highlighting}[]
\NormalTok{test\_vector }\OtherTok{\textless{}{-}} \FunctionTok{c}\NormalTok{(}\DecValTok{1}\NormalTok{,}\DecValTok{5}\NormalTok{,}\DecValTok{11}\SpecialCharTok{:}\DecValTok{22}\NormalTok{)}

\NormalTok{(x }\SpecialCharTok{\%in\%}\NormalTok{ test\_vector }\SpecialCharTok{|}\NormalTok{ y }\SpecialCharTok{\%in\%}\NormalTok{ test\_vector }\SpecialCharTok{|}\NormalTok{ z }\SpecialCharTok{\%in\%}\NormalTok{ test\_vector)}
\end{Highlighting}
\end{Shaded}

\begin{verbatim}
## [1] TRUE
\end{verbatim}

\section{Exercise 6}\label{exercise-6}

Show which value is contained in the test vector. To do this you will
need to create an element-wise logical vector using operators.
\texttt{x\ ==\ vector}. Once you have done that you will need to use
slicing to return all indices that have matches. \textbf{Note: your
output should be two integers}

\begin{Shaded}
\begin{Highlighting}[]
\CommentTok{\#One way to check which of x, y, or z is in the test vector with a boolean}
\NormalTok{sample\_vector }\OtherTok{\textless{}{-}} \FunctionTok{c}\NormalTok{(x, y,z)}

\NormalTok{locate\_test }\OtherTok{\textless{}{-}} \ControlFlowTok{function}\NormalTok{ (n) \{}
\NormalTok{  n }\SpecialCharTok{\%in\%}\NormalTok{ test\_vector}
\NormalTok{\}}

\FunctionTok{locate\_test}\NormalTok{(sample\_vector)}
\end{Highlighting}
\end{Shaded}

\begin{verbatim}
## [1] FALSE  TRUE  TRUE
\end{verbatim}

\begin{Shaded}
\begin{Highlighting}[]
\CommentTok{\#With slicing}
\NormalTok{test\_vector[x }\SpecialCharTok{==}\NormalTok{ test\_vector }\SpecialCharTok{|}\NormalTok{ y }\SpecialCharTok{==}\NormalTok{ test\_vector }\SpecialCharTok{|}\NormalTok{ z }\SpecialCharTok{==}\NormalTok{ test\_vector]}
\end{Highlighting}
\end{Shaded}

\begin{verbatim}
## [1]  5 20
\end{verbatim}

\begin{Shaded}
\begin{Highlighting}[]
\NormalTok{test\_vector[}\SpecialCharTok{{-}}\DecValTok{1}\NormalTok{]}
\end{Highlighting}
\end{Shaded}

\begin{verbatim}
##  [1]  5 11 12 13 14 15 16 17 18 19 20 21 22
\end{verbatim}

\begin{Shaded}
\begin{Highlighting}[]
\NormalTok{y }\OtherTok{\textless{}{-}} \DecValTok{1}

\NormalTok{f }\OtherTok{\textless{}{-}} \ControlFlowTok{function}\NormalTok{(x) \{}
\NormalTok{  y }\OtherTok{\textless{}{-}} \DecValTok{2}
\NormalTok{  y}\SpecialCharTok{\^{}}\DecValTok{2} \SpecialCharTok{+} \FunctionTok{g}\NormalTok{(x)}
\NormalTok{\}}

\NormalTok{g }\OtherTok{\textless{}{-}} \ControlFlowTok{function}\NormalTok{(x) \{}
\NormalTok{  x }\SpecialCharTok{*}\NormalTok{ y}
\NormalTok{\}}

\FunctionTok{f}\NormalTok{(}\DecValTok{10}\NormalTok{)}
\end{Highlighting}
\end{Shaded}

\begin{verbatim}
## [1] 14
\end{verbatim}

\end{document}
